%Template based off AMS Dept HW class file v0.05 by Eric Harley
\documentclass{article}

\usepackage{verbatim}
\usepackage{graphicx}
\usepackage{amssymb,amsmath,amsthm,amsfonts,mathtools}
\usepackage{hyperref}
\usepackage{fullpage}
\usepackage{url}
\usepackage{listings}
\usepackage[utf8]{inputenc}
\usepackage{boxedminipage}
\usepackage{enumerate}

\newcommand{\R}{\mathbb{R}}
\newcommand{\Z}{\mathbb{Z}}
\newcommand{\N}{\mathbb{N}}

\newenvironment{problem}{%
  \bigskip\noindent\begin{boxedminipage}{\columnwidth}%
}{\end{boxedminipage}}

\newenvironment{solution}{%
  \begin{trivlist}\item[]%
}{%
  \mbox{}\penalty10000\hfill\ensuremath{\scriptscriptstyle\blacksquare}%
  \end{trivlist}%
}

\begin{document}

\begin{flushright}
Your name\\
CS 261, Spring 2020\\
Homework 4\\
Due Monday, April 27
\end{flushright}

\begin{problem}
\textbf{1.} Suppose that we want to maintain a random sample of a cash-register-model data stream whose size depends on the number of stream elements seen so far: after seeing $n$ elements, we should have a sample of size $\lceil\log_2 n\rceil$, chosen uniformly at random among all possible samples of that size. However, we have no advance knowledge of how big $n$ is going to be.

\smallskip
Explain why, in this scenario, it is not possible to maintain a sketch that allows us to generate samples of the desired size (with the exact probability distribution specified above) using a sublinear amount of space.
\end{problem}

\begin{solution}
% YOUR SOLUTION TO PROBLEM 1 GOES HERE
\end{solution}

\begin{problem}
\textbf{2.} Suppose we have two Boyer--Moore majority sketches $(m_A,c_A)$ and $(m_B,c_B)$ for input sequences $A$ and $B$. Explain how to use them to compute a single sketch for the concatenation of sequences $AB$, again consisting of a pair of numbers $(m,c)$. Your combined sketch does not have to be equal to the sketch that the majority algorithm would produce for $AB$, but it should provide an estimate for the number $\operatorname{count}(x)$ of occurrences of each element $x$ that is bounded between $\operatorname{count}(x)-|AB|/2$ and  $\operatorname{count}(x)$, just like the majority algorithm would. (In particular, this implies that if $AB$ has a majority element, your combination will choose that element as its value of $m$). Explain why your combination has this property.
\end{problem}

\begin{solution}
% YOUR SOLUTION TO PROBLEM 2 GOES HERE
\end{solution}

\begin{problem}
\textbf{3.} Suppose that we are maintaining a MinHash sketch of size $k$ for a cash-register-model data stream, and that we additionally store one more piece of information, the \emph{largest} hash value among the $k$ values already stored in the stream.

\smallskip\textbf{(a)} What is the probability that the $n$th item in the stream has a hash value smaller than this largest value? You can assume that all items in the stream have distinct hash values and that the hash function is uniformly random.

\smallskip\textbf{(b)} Suppose that the algorithm for maintaining the sketch, when each new value $x$ is processed, does the following steps. It first checks in constant time whether $x$ has a smaller hash value than the stored largest hash value. If $x$ does have a smaller hash value, the algorithm updates the sketch in $O(k)$ time, but if not the algorithm discards $x$ in constant time. What is the expected total time for this procedure to process a sequence of $n$ values? State your answer using $O$-notation, as a function of $k$ and $n$.

\smallskip
(Note: faster algorithms for updating the sketch are possible. Please answer this question using the stated time for an update, rather than using these faster algorithms.)
\end{problem}

\begin{solution}
% YOUR SOLUTION TO PROBLEM 3 GOES HERE
\end{solution}

\begin{problem}
\textbf{4.}  The lecture notes sketch a method for estimating the number of set elements in any query interval $[\ell,r]$, in the turnstile model. The method uses logarithmically many count-min sketches, for sets $S_i$ generated from the given data set $S$ by rounding each element of $S$ down to a multiple of $2^i$.

\smallskip
Provide more detailed pseudocode for a subroutine used in this method that takes as input the pair $(\ell,r)$ and produces as output a sequence of pairs $(i,x)$ of elements to query in the count-min sketch for $S_i$. Your subroutine's output should have the property that for each number $y$ in the range $[\ell,r]$, there should be exactly one pair $(i,x)$ in the output such that rounding $y$ down to a multiple of $2^i$ produces $x$.

\smallskip
For instance, for the range $[3,11]$ your output should be the set of three pairs $(0,3)$, $(2,4)$, and $(2,8)$ (in any order). Every number in the range $[3,11]$ rounds to one of these pairs: $3$ rounded to a multiple of $2^0$ gives $3$, matching the pair $(0,3)$; $4$, $5$, $6$, and $7$ rounded to a multiple of $2^2$ give $4$, matching the pair $(2,4)$, and 8, 9, 10, and 11 rounded to a multiple of $2^2$ give 8, matching the pair $(2,8)$.
\end{problem}

\begin{solution}
% YOUR SOLUTION TO PROBLEM 4 GOES HERE
\end{solution}

\end{document}
